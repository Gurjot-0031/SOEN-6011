\documentclass[12pt]{article}
\usepackage[utf8]{inputenc}
\usepackage{algorithm}
\usepackage{algorithmic}
\usepackage{mathtools}
\usepackage{amsmath}
\usepackage{mathptmx}
\usepackage{geometry}
\usepackage{biblatex}
\addbibresource{mybibliography.bib}

\title{Deliverable 1}
\author{Gurjot Singh}
\date{Student ID: 40091208}

\begin{document}

\maketitle

\section*{Problem 1:}
\textbf{Give a brief description, not exceeding one page, of your function, including the domain
and co-domain of function, and the characteristics that make it unique. }
\vskip 1 cm
The function $f(x,y) = x^y$ is also called power function in the field of Mathematics. It refers to raising x to the power of y. Here, x is the base and y is the exponent to which x is raised to.\cite{r2}
\paragraph{DOMAIN\vskip 0.5cm}
    All real numbers.
\paragraph{CO-DOMAIN\vskip 0.5cm}
    All real numbers.
\paragraph{Characteristics of the function}
    
    
    \begin{enumerate}
\item As the power increases, the graph of power function flattens somewhat near the origin and become steeper away from the origin.
\item The function is \textbf{not a one to one} function. For Example:- both $2^2$ and $-2^2$ evaluates to the same result $(4)$.
\item However, this function is an \textbf{onto function}.
\end{enumerate}


\section*{Problem 2:}
\textbf{Express requirements of your function based on the style given in the ISO/IEC/IEEE
29148 Standard. Associate each requirement with a unique identifier.}
\vskip 1 cm
\textbf{Functional requirements for function : $f(x,y) = x^y$}
\vskip 1cm
\textbf{First Requirement}
\vskip 0.25cm
ID - R1
\vskip 0.25cm
Type - Functional Requirement
\vskip 0.25cm
Version - 1.0
\vskip 0.25cm
Difficulty - Intermediate
\vskip 0.25cm
Description - The function calculates the result and displays in appropriate numeric format.
\vskip 1cm
\textbf{Second Requirement}
\vskip 0.25cm
ID - R2
\vskip 0.25cm
Type - Functional Requirement
\vskip 0.25cm
Version - 1.0
\vskip 0.25cm
Difficulty - Easy
\vskip 0.25cm
Description - If the inputs are provided in a format other than numeric values, it should throw an error.
\vskip 1cm

\textbf{Third Requirement}
\vskip 0.25cm
ID - R3
\vskip 0.25cm
Type - Functional Requirement
\vskip 0.25cm
Version - 1.0
\vskip 0.25cm
Difficulty - Easy
\vskip 0.25cm
Description - The function validates the input values. These should fall within the domain of the function (i.e. Real numbers).
\vskip 1cm

\textbf{Fourth Requirement}
\vskip 0.25cm
ID - R4
\vskip 0.25cm
Type - Functional Requirement
\vskip 0.25cm
Version - 1.0
\vskip 0.25cm
Difficulty - Easy
\vskip 0.25cm
Description - If the value of y is a decimal number, the value of x should be 0 or greater than 0.
\vskip 0.25cm


\section*{Problem 3:}
\textbf{Give a brief description of your algorithms and express each of them in pseudocode.} 
\vskip 1 cm

Below are the two algorithms which were considered for calculating the power function.

\textbf{Algorithm A:} It involves an recursive approach for calculating the powers of integer values of $x$ and $y$. The recursion is further made smart by checking whether the power is even or odd, and finding the power accordingly, in $O(log(y))$ time, which is very efficient with respect to the approach used in algorithm B.
For calculating decimal powers of a number, this algorithm breaks the problem into two sub-problems:-
\begin{enumerate}
    \item Calculating the power of \textbf{Integral-Part} of $y$  .
    \item Calculating the power of \textbf{Fractional-Part} of $y$  .
\end{enumerate}
Problem 1 is solved as described above, but for problem 2, the decimal number is first converted into fraction. The denominator part in this fraction denotes the Nth root of $x$. After calculating the Nth root of $x$, the Nth root is then raised to the Numerator of that fraction.

\vskip 0.25cm
After both these sub-problems are solved, the results of problem 1 and 2 are multiplied together and we get our answer.
\vskip 0.25cm
\textbf{Advantage of Algorithm A:}
\vskip 0.25cm
 Its amortized time complexity is $O(log(y))$. Hence, it is very efficient with respect to the normal way of finding a power by basic multiplication. Moreover, recursion is easy to understand and has high readability.
 
 \vskip 0.25cm
\textbf{Disadvantage of Algorithm A:}
\vskip 0.25cm
 It is implementation is bit complex.

\vskip 1cm
\textbf{Algorithm B:} It involves the conventional approach for calculating the powers by using iterative multiplication.\cite{r1} Its time complexity is $O(y)$, which is not very efficient with respect to the approach used in algorithm A.
\vskip 0.25cm
\textbf{Advantage of Algorithm B:}
\vskip 0.25cm
 It is pretty straightforward to implement as compared to the algorithm A. Also, it avoids memory overflow of input.
 
 \vskip 0.25cm
\textbf{Disadvantage of Algorithm B:}
\vskip 0.25cm
 Its time complexity is $O(y)$. Therefore, for large values of y, it takes a lot of time to execute. Hence, it is not that efficient as compared to the algorithm A.
 
\vskip .2cm
After considering the advantages and disadvantages of both A and B, I selected the \textbf{algorithm A}.
\vskip 1cm

\section*{Pseudocode for Algorithm A}
Calculate: $f(x,y) = x^y$

\begin{algorithm}
\caption{POWER(x,y)}
\begin{algorithmic} 
 \item \textbf{IsADecimalNumber(a):} It checks whether a is not an integer.
 \vskip .06cm
 \textbf{IsAnInteger(a):} It checks whether a is an integer.
 \vskip .06cm
 \textbf{GetNR(a):} It returns the Numerator of the decimal number(a) after converting it into fraction.
 \vskip .06cm
 \textbf{GetDR(a):} It returns the Denominator of the decimal number(a) after converting it into fraction.
 \vskip .06cm
 \textbf{GetNthPower(a,n):} It returns the nth root of integer a.
 
\STATE 1. $Input(x,y)$\\
\STATE 2. if $x = 0$ then\\
\STATE 3. \qquad if $y \not=0$ then\\
\STATE 4. \qquad \qquad return $0$
\STATE 5. \qquad else
\STATE 6. \qquad \qquad return UNDEFINED
\STATE 7. \qquad end if
\STATE 8. end if

\STATE 9. if $y = 0$ then\\
\STATE 10. \qquad if $x \not=0$ then\\
\STATE 11. \qquad \qquad return $1$
\STATE 12. \qquad else
\STATE 13. \qquad \qquad return UNDEFINED
\STATE 14. \qquad end if
\STATE 15. end if

\STATE 16. if $x < 0$ then\\
\STATE 17. \qquad if $IsADecimalNumber(y) $ then\\
\STATE 18. \qquad \qquad return ERROR
\STATE 19. \qquad end if
\STATE 20. end if

\STATE 21. if $IsAnInteger(y)$ then\\
\STATE 22.\qquad return $POWER(x,y)$
\STATE 23. else
\STATE 24.\qquad  $num \leftarrow GetNR(DecimalPart(y))$
\STATE 25.\qquad  $den \leftarrow GetDR(DecimalPart(y))$
\STATE 26.\qquad  $NthRoot \leftarrow GetNthPower(x,denominator)$
\STATE 27.\qquad  return $POWER(x,IntegralPart(y))*POWER(NthRoot,num)$
\STATE 28. end if
\end{algorithmic}
\end{algorithm}


\newpage
\vskip 1cm

\section*{Pseudocode for Algorithm B}
Calculate: $f(x,y) = x^y$

\begin{algorithm}
\caption{POWER}
\begin{algorithmic} 
\STATE 1. $Input(x,y)$\\
\STATE 2. $output \leftarrow 1 $
\STATE 3. while $y \not=0 $ then\\
\STATE 4.\qquad  $output \leftarrow output*x$
\STATE 5.\qquad  $y \leftarrow y-1$
\STATE 6. end while
\STATE 7. return output
\end{algorithmic}
\end{algorithm}

\printbibliography
\end{document}
